\section{Analysis}
Below is the Output of each program processing 20 gibberish text:
	\begin{table}[h]
	    \centering
	    \begin{tabular}{r|c|c|c|c}
	       \# & Arith. encoding& Arith. decoding &  Huff. encoding & Huff. decoding\\
       \hline
	       1& 0.000138 & 0.000402 &0.000264& 0.001167\\
	       2 &0.000144& 0.000134 &0.000189& 0.001261\\
	       3 &0.000296 & 0.000120 &0.000257& 0.001261\\
	       4 &0.000110& 0.000274 &0.000204& 0.001187\\
	       5 &0.000109 &0.000167  &0.000193& 0.001190\\
	       6 &0.000117& 0.000137  &0.000250&0.001237\\
	       7 &0.000233 & 0.000333 &0.000186&0.001431 \\
	       8 &  0.000111&0.000278& 0.000242&0.001448\\
	       9  & 0.000237  &0.000140&0.000412&0.001342\\
	       10 & 0.000241 &0.000283&0.000490&0.001317 \\
	       11  &0.000228 &0.000145& 0.000285&0.001266\\
	       12 & 0.000111 &0.000278& 0.000287&0.001603\\
	       13  & 0.000116 &0.000298&0.000281 &0.001161\\
	       14 & 0.000107 &0.000286&0.000244& 0.001496\\
	       15  &0.000236  &0.000288& 0.000294&0.001551\\
	       16 & 0.000231 &0.000114& 0.000340&0.001314\\
	       17  & 0.000091 &0.000107& 0.000331&0.001679\\
	       18 &  0.000246&0.000104&0.000295 &0.001391\\
	       19  &  0.000090 &0.000131&0.000240&0.001122\\
	       20 & 0.000263 &0.000268& 0.000252& 0.001237\\
	       \hline
	       Avg&0.000172 & 0.000206&0.000265& 0.001333
	    \end{tabular}
	    \caption{Caption}
	    \label{tab:my_label}
	\end{table}
	From the table, we can see that the execution time of Huffman coding in our selected repository is longer, which breaks down to the sections we will discuss below.
	\subsection{I/O File difference}
	In arithmetic encoding, the output of a string could be perceived as a binary float number, which represent each character of the string. The program only needs to parse each bit and restore the bits of the string based on encoded file.
	\\
	However, in Huffman encoding/decoding, the program has to decode the file based on an external chart, which represents the encoded bits of each character. In this way, the program will take more time making file access, and thus slows the execution time of file parsing. 
	\\