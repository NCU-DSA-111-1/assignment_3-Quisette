\documentclass[11pt,a4paper]{report}
\usepackage[utf8]{inputenc}
\usepackage[T1]{fontenc}
\usepackage{amsmath}
\usepackage{amsfonts}
\usepackage{amssymb}
\usepackage{graphicx}





\title{A simple \LaTeX\ Tutorial and Note}
\author{Quisette Chung}


% this preamble is used for document class:article.
%\usepackage{xeCJK}
\usepackage{graphicx}
\usepackage{float}
\graphicspath{{image/}}
\usepackage{graphpap} % 方格紙


\usepackage[most]{tcolorbox}
\usepackage{cleveref}
%\setCJKmainfont{Noto Serif TC}

\tcbset{theostyle/.style={
		enhanced,
		sharp corners,
		attach boxed title to top left={
			xshift=-1mm,
			yshift=-4mm,
			yshifttext=-1mm
		},
		top=1.5ex,
		colback=white,
		colframe=blue!75!black,
		fonttitle=\bfseries,
		boxed title style={
			sharp corners,
			size=small,
			colback=blue!75!black,
			colframe=blue!75!black,
		} 
}}
\tcbset{defstyle/.style={
		enhanced,
		sharp corners,
		attach boxed title to top left={
			xshift=-1mm,
			yshift=-4mm,
			yshifttext=-1mm
		},
		top=1.5ex,
		colback=white,
		colframe=green!50!black,
		fonttitle=\bfseries,
		boxed title style={
			sharp corners,
			size=small,
			colback=green!50!black,
			colframe=green!50!black,
		} 
}}
\tcbset{expstyle/.style={
		enhanced,
		sharp corners,
		attach boxed title to top left={
			xshift=-1mm,
			yshift=-4mm,
			yshifttext=-1mm
		},
		top=1.5ex,
		colback=white,
		colframe=red!70!black,
		fonttitle=\bfseries,
		boxed title style={
			sharp corners,
			size=small,
			colback=red!70!black,
			colframe=red!70!black,
		} 
}}

\newtcbtheorem[number within=section]{Thm}{Theorem}{
	theostyle
}{thm}

\newtcbtheorem[number within=section]{Def}{Definition}{
	defstyle
}{def}

\newtcbtheorem[number within=section]{Exp}{Example}{
	expstyle
}{exp}

\newtcbtheorem[number within = tcb@cnt@Thm]{Lem}{Lemma}{
	theostyle
}{lem}
\newtcbtheorem[number within=tcb@cnt@Thm]{Cor}{Corollary}{
	theostyle
}{cor}

\newtcbtheorem[number within=section]{Wng}{Warning}{
	expstyle
}{Wng}

\newtcbtheorem[number within=section]{Tip}{Tips}{
	theostyle
}{Tip}

\begin{document}
	\maketitle
	\tableofcontents
	\chapter{Test Chapter}
		\begin{Thm}{Test Theorem}{}
			This is the content of The Test Theorem.
		\end{Thm}
		\begin{Lem}{Test Lemma}{}
			This is the content of The Test Lemma.
		\end{Lem}
		\begin{Cor}{Test Corollary}{}
			This is the content of The Test Corollary.
		\end{Cor}
		\begin{Def}{Test Definition}{}
			This is the content of The Test Definition.
		\end{Def}
		\begin{Exp}{The Test Example}{}
		 This is the content of The Test Example.
		\end{Exp}
	\section{Tables}
		\begin{table}[H]
			\caption{實驗C-2之原始數據}
			\centering
			\begin{tabular}[c]{ccccc}
				
				\textbf{滑車質量}& \textbf{外加砝碼} & \textbf{初速}& \textbf{末速} & \textbf{加速度}  \\
				$m_1$ (g)  &  $m_2$ (g)&$v_i$ (cm/s) & $v_f$ (cm/s) & $a$ (cm/s$^2$)\\
				\hline \\
				233.0 & 10.0& 49.75&80.64 &  50.3\\							
				223.0 & 20.0& 71.94&104.1 & 70.8 \\
				213.0 & 30.0& 90.90& 131.5&  112.9 \\
				203.0 & 40.0&100.0 &161.2 & 199.8 \\
				193.0 & 50.0& 140.8&212.7 & 317.7 \\
			\end{tabular}
		\end{table}
	
	\section{Pictures}
		\begin{figure}[H]
			\centering 
			\includegraphics[width=0.5\textwidth]{figure1.jpg}
			\caption{滑車受外加砝碼牽引作用之系統示意圖\cite{ref1}}
		\end{figure}
	\section{Equations}
	This is a sample Equation. We can reference like this one \ref{sampleeq} and this one \eqref{sampleeq}
		\begin{equation}
			\label{sampleeq}
			1+2 = 3
		\end{equation}
		\begin{equation}
			\frac{d}{dx}f(x) = x^2 +e^{i\pi}
		\end{equation}
	
	
	\begin{thebibliography}{99}  
		
		\bibitem{ref1}實驗二:運動軌跡與相關物理量測量,國立清華大學普通物理實驗室(http://www.phys.nthu.edu.tw/~gplab/exp002.html)
		\bibitem{ref2}普通物理實驗講義,國立中央大學物理學系
		%\bibitem{ref3}
		%\bibitem{ref4}
		%\bibitem{ref5}
		
	\end{thebibliography} 
\end{document}
	
